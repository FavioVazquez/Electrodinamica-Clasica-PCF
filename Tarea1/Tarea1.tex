\documentclass[a4paper,10pt]{article}
\usepackage[utf8]{inputenc}
\usepackage[spanish]{babel}
\usepackage[affil-it]{authblk}
\usepackage{enumerate}
\usepackage{graphicx}
\usepackage{hyperref}
\usepackage{amsmath}
\usepackage{amssymb}
\usepackage{cancel}
\usepackage[usenames, dvipsnames]{color}
\usepackage{tikz}
\usepackage[labelfont=bf]{caption}
\usepackage{subcaption} %Multiple images
\usepackage{multicol} % Multiple columns
\usepackage{float}
\usepackage{cleveref}
 \usepackage{relsize} % bigger math symbols
\usepackage[margin=1.4in]{geometry}
\usepackage[titletoc,toc,title]{appendix}
\usepackage{enumitem}
\usepackage{etoolbox}
\usepackage{mdframed} %frame theorems
\usetikzlibrary{calc}
\numberwithin{equation}{section}

% Enviroment for theorems
\newmdtheoremenv[frametitle=Teorema]{theo}{Theorem}

% Circled words
\newcommand{\circled}[2][]{%
  \tikz[baseline=(char.base)]{%
    \node[shape = circle, draw, inner sep = 1pt]
    (char) {\phantom{\ifblank{#1}{#2}{#1}}};%
    \node at (char.center) {\makebox[0pt][c]{#2}};}}
\robustify{\circled}

%Appendices in spanish
\renewcommand{\appendixname}{Ap\'endices}
\renewcommand{\appendixtocname}{Ap\'endices}
\renewcommand{\appendixpagename}{Ap\'endices}

%Zero delimiter
\newcommand{\zerodel}{.\kern-\nulldelimiterspace}

%Columns separation
\setlength{\columnsep}{1cm}

%Indentation
\setlength{\parindent}{0ex}

%Multiple References

\crefrangelabelformat{equation}{(#3#1#4--#5\crefstripprefix{#1}{#2}#6)}

\usepackage{xparse}

%Boxes

\newcommand*{\boxcolor}{blue}
\makeatletter
\renewcommand{\boxed}[1]{\textcolor{\boxcolor}{%
\tikz[baseline={([yshift=-1ex]current bounding box.center)}] \node [rectangle, minimum width=1ex,rounded corners,draw] {\normalcolor\m@th$\displaystyle#1$};}}
 \makeatother

%Constantes
\newcommand{\euler}{\mathrm{e}}
\newcommand{\im}{i}

%Lemas, teoremas, definiciones y pruebas
\newcommand{\definicion}{\textbf{Definición: }}
\newcommand{\lema}{\textbf{Lema: }}
\newcommand{\teorema}{\textbf{Teorema: }}
\newcommand{\prueba}{\textbf{Prueba: }}
\newcommand{\proposicion}{\textbf{Proposición: }}
\newcommand{\corolario}{\textbf{Corolario: }}

% Definición de las secciones y su numeración

\makeatletter
\def\@seccntformat#1{%
  \expandafter\ifx\csname c@#1\endcsname\c@section\else
  \csname the#1\endcsname\quad
  \fi}
\makeatother

%opening
\title{Electrodinámica Clásica. Tarea \# 1}
\author{Favio Vázquez\thanks{Correo: favio.vazquezp@gmail.com}}\affil{Instituto de Ciencias Nucleares. Universidad Nacional Autónoma de México.}
\date{}

\begin{document}

\makeatletter
\def\@maketitle{%
  \newpage
  \null
  \vskip 2em%
  \begin{center}%
  \let \footnote \thanks
    {\Large\bfseries \@title \par}%
    \vskip 1.5em%
    {\normalsize
      \lineskip .5em%
      \begin{tabular}[t]{c}%
        \@author
      \end{tabular}\par}%
    \vskip 1em%
    {\normalsize \@date}%
  \end{center}%
  \par
  \vskip 1.5em}
\makeatother

\maketitle

\section{Problema 1. Problema 1.2 de Classical Electrodynamics (3ra ed) de 
Jackson \cite{jackson}.}

La función delta de Dirac en tres dimensiones puede tomarse como el límite impropio 
mientras $\alpha \rightarrow 0$ de la función gaussiana 

$$
D(\alpha;x,y,z) = (2\pi)^{-3/2}\alpha^{-3}\exp\left[-\frac{1}{2a}(x^2 + y^2 + z^2 \right].
$$

Considere un sistemas de coordenadas ortogonal general especificado por las superficies 
$u = \text{constante}$, $v = \text{constante}$, $w = \text{constante}$, con elementos 
de longitud $du/U$, $dv/V$, $dw/W$ en las tres direcciones perpendiculares. Mostrar 
que

$$
\delta(\mathbf{x} - \mathbf{x}') = \delta(u - u')\delta(v - v')\delta(w - w')\cdot UVW
$$

considerando el límite de la gaussiana arriba. Note que mientras $\alpha \rightarrow 
0$, sólo el elemento de longitud infinitesimal necesita ser usado para la distancia e
entre los puntos en la exponencial.

\vspace{.3cm}

\underline{Solución:} \vspace{.3cm}

Partiendo de la ecuación 

\begin{equation}
 D(\alpha;x,y,z) = (2\pi)^{-3/2}\alpha^{-3}
 \exp\left[-\frac{1}{2a}(x^2 + y^2 + z^2 \right],
\end{equation}

y haciendo el cambio de variable $x \rightarrow x-x'$, $y \rightarrow y-y'$, 
$z \rightarrow z-z'$ obtenemos 

\begin{equation*}
 D(\alpha;x-x',y-y',z-z') = (2\pi)^{-3/2}\alpha^{-3}
 \exp\left\{-\frac{1}{2a}[(x-x')^2 + (y-y)^2 + (z-z')^2] \right\}.
\end{equation*}

Notamos que mientras $\alpha \rightarrow 0 \Rightarrow D \rightarrow 0$, a 
menos que  $x \rightarrow x-x'$, $y \rightarrow y-y'$, $z \rightarrow z-z' 
\rightarrow 0$ también. Y recordando que $x-x' \rightarrow 0 \Rightarrow dx$, etc., 
tenemos 

\begin{equation}
 D(\alpha;x-x',y-y',z-z') = (2\pi)^{-3/2}\alpha^{-3}
 \exp\left[-\frac{1}{2a}(dx^2 + dy^2 + dz^2) \right].
\end{equation}

Recordamos también que el elemento de longitud se escribe como, 

\begin{equation}
 ds^2 = dx^2 + dy^2 + dz^2,
\end{equation}

entonces 

\begin{equation}
 D(\alpha;x-x',y-y',z-z') = (2\pi)^{-3/2}\alpha^{-3}
 \exp\left[-\frac{1}{2a}(ds^2) \right].
 \label{eq:15}
\end{equation}

Y utilizando el sistema de coordenadas que hay que considerar según el problema, 
podemos reescribir $ds^2$ como 

\begin{equation}
 ds^2 = \left(\frac{du}{U}\right) + \left(\frac{dv}{V}\right) + 
 \left(\frac{dw}{W}\right),
\end{equation}

donde recalcamos que no estamos diciendo que $dx = du/U$, etc., sino que estamos 
usando el hecho de que $ds^2$ es el mismo en todos los sistemas ortogonales, 
por lo tanto podemos reescribir \eqref{eq:15} como 

\begin{equation}
 D(\alpha;u,v,w) = (2\pi)^{-3/2}\alpha^{-3}
 \exp\left[-\frac{1}{2\alpha^2}\left(\frac{du^2}{U^2} + 
 \frac{dv^2}{V^2} + \frac{dw^2}{W^2}\right) \right].
\end{equation}

Ahora expandimos de vuelta los diferenciales en diferencias, 

\begin{equation*}
 D(\alpha;u-u',v-v',w-w') = (2\pi)^{-3/2}\alpha^{-3}
 \exp\left\{-\frac{1}{2\alpha^2}\left[\frac{(u - u')^2}{U^2} + 
 \frac{(v-v')^2}{V^2} + \frac{(w-w')^2}{W^2}\right] \right\},
\end{equation*}

que podemos escribir como (utilizando el hecho que $\exp(a+b) = \exp(a)\exp(b)$),

\begin{equation*}
 D = (2\pi)^{-3/2}\alpha^{-3}
 \exp\left[-\frac{1}{2\alpha^2}\frac{(u - u')^2}{U^2}\right]
 \exp\left[-\frac{1}{2\alpha^2}\frac{(v-v')^2}{V^2}\right]
 \exp\left[-\frac{1}{2\alpha^2}\frac{(w-w')^2}{W^2}\right],
\end{equation*}

y esta ecuación puede expresarse de una forma más conveniente como 

\begin{equation}
 D = \frac{\exp\left[-\frac{1}{2\alpha^2}\frac{(u - u')^2}{U^2}\right]}{\sqrt{2\pi}\alpha}
 \frac{\exp\left[-\frac{1}{2\alpha^2}\frac{(v-v')^2}{V^2}\right]}{\sqrt{2\pi}\alpha}
 \frac{\exp\left[-\frac{1}{2\alpha^2}\frac{(w-w')^2}{W^2}\right]}{\sqrt{2\pi}\alpha}.
\label{eq:110}
\end{equation}

La cual es una ecuación que comienza a parecerse a la definición de la delta. Ahora 
si reemplazamos en cada término del lado derecho de \eqref{eq:110} las $\alpha$'s 
por $\alpha_u/U$, $\alpha_v/V$ y $\alpha_w/W$ respectivamente\footnote{Podemos 
hacer esto si $\alpha_u,\alpha_v,\alpha_w \rightarrow 0$ mientras que 
$\alpha \rightarrow 0$.} y obtenemos 

\begin{equation}
 D(\alpha;u-u',v-v',w-w') = \frac{\exp\left[-\frac{(u - u')^2}{2\alpha_u}\right]}{\sqrt{2\pi}\alpha}
 \frac{\exp\left[\frac{(v-v')^2}{2\alpha_v}\right]}{\sqrt{2\pi}\alpha}
 \frac{\exp\left[\frac{(w-w')^2}{2\alpha_w}\right]}{\sqrt{2\pi}\alpha}\cdot UVW.
 \label{eq:112}
\end{equation}

Ahora tomando el límita mientras $\alpha_u,\alpha_v,\alpha_w \rightarrow 0$, 
vemos que cada término del lado derecho de \eqref{eq:112} se convierte en un 
delta de Dirac unidimensional y el lado izquierdo entonces en la expresión 
general de la delta de Dirac tridimensional, por lo tanto \eqref{eq:112} se 
transforma en 

\begin{equation*}
 \boxed{\underset{\alpha_u,\alpha_v,\alpha_w \rightarrow 0}{\text{lím}} 
 D(\alpha;u-u',v-v',w-w') = \delta(\mathbf{x} - \mathbf{x}') = 
 \delta(u - u')\delta(v - v')\delta(w - w')\cdot UVW.}
\end{equation*}

\section{Problema 2. Problema 1.3 de Classical Electrodynamics (3ra ed) de 
Jackson \cite{jackson}.}

Usando las funciones delta de Dirac en las coordenadas apropiadas, exprese las 
siguientes distribuciones de carga como las densidades de carga tridimensionales 
$\rho(\mathbf{b}).$

\begin{enumerate}[label=\textbf{(\alph*)}]
 \item En coordenadas esféricas, una carga $Q$ uniformemente distribuida sobre 
 una concha esférica de radio $R$.
 \item En coordenadas cilíndricas, una carga $\lambda$ por unidad de longitud 
 uniformemente distribuida sobre una superficie cilíndrica de radio $b$.
 \item En coordenadas cilíndricas, una carga Q extendida uniformemente sobre 
 un disco plano de grosor despreciable y radio $R$.
 \item Lo mismo que en la parte \textbf{(c)}, pero usando coordenadas esféricas.
\end{enumerate}

\vspace{.3cm}

\underline{Solución:} \vspace{.3cm}

\section{Problema 3. Problema 1.4 de Classical Electrodynamics (3ra ed) de 
Jackson \cite{jackson}.}

Cada una de las esferas cargadas de radio $a$, una conductora, una con una densidad 
de carga uniforma adentro de su volumen, y una con una densidad de carga esféricamente 
simétrica que varía radialmente como $r^n$ ($n > -3)$, tiene una carga total $Q$. 
Use el teorema de Gauss para obtener los campos eléctricos tanto adentro como afuera 
de la esfera. Esboce el comportamiento de los campos como una función del radio 
para las primeras dos esfera, y para la tercera com $n= -2,+2$. 

\vspace{.3cm}

\underline{Solución:} \vspace{.3cm}

\section{Problema 4. Problema 1.5 de Classical Electrodynamics (3ra ed) de 
Jackson \cite{jackson}.}

El potencial promedio en el tiempo de un átomo de hidrógeno neutro está dado por 

$$
\Phi = \frac{q}{4\pi\epsilon_0}\frac{\euler^{-\alpha r}}{r}\left(1 + 
\frac{\alpha r}{2}\right)
$$

donde $q$ es la magnitud de la carga electrónica, y $\alpha^{-1} = a_0/2$, siendo 
$a_0$ el radio de Bohr. Encuentre la distribución de carga (tanto continua como 
discreta) que dará este potencial e interpreta tu resultado físicamente.

\vspace{.3cm}

\underline{Solución:} \vspace{.3cm}

\section{Problema 5. Problema 1.6 de Classical Electrodynamics (3ra ed) de 
Jackson \cite{jackson}.}

Un capacitor simple es un dispositivo formado por dos conductores aislados 
adyacentes el uno del otro. Si cargas iguales y opuestas se colocan en los 
conductores, habrá una cierta diferencia de potencial entre ellos. La razón de 
la magnitud de la carga en un conductor a la magnitud de la diferencia de 
potencial es llamada capacitancia (en unidades SI se mide en faradios). Usando la 
ley de Gauss, calcule la capacitancia de 

\begin{enumerate}[label=\textbf{(\alph*)}]
 \item dos láminas grandes, planas, de área $A$, separadas por una pequeña 
 distancia $d$;
 \item dos esferas concéntricas conductoras con radios $a,b$ $(b > a)$;
 \item dos cilindros concéntricos conductores de longitud $L$, grande comparada 
 a sus radios $a,b$ $(b > a)$.
 \item ¿Cuál es el diámetro interno del conductor externo en un cable coaxial 
 lleno de aire, cuyo conductor central es un alambre cilíndrico de diámetro 
 1 mm y cuya capacitancia es $3\times 10^{-11}$ F/m? ¿$3\times 10^{-12}$ F/m?
\end{enumerate}

\vspace{.3cm}

\underline{Solución:} \vspace{.3cm}

\begin{thebibliography}{10}
\bibitem{jackson}
J. Jackson, \emph{Classical Electrodynamics}, 3ra edición. John Wiley and Sons, Inc. 
1999.
\end{thebibliography}


\end{document}