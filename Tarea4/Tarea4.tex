\documentclass[a4paper,11pt]{article}
\usepackage[utf8]{inputenc}
\usepackage[spanish]{babel}
\usepackage[affil-it]{authblk}
\usepackage{enumerate}
\usepackage{graphicx}
\usepackage{hyperref}
\usepackage{amsmath}
\usepackage{amssymb}
\usepackage{cancel}
\usepackage[usenames, dvipsnames]{color}
\usepackage{tikz}
\usepackage[labelfont=bf]{caption}
\usepackage{subcaption} %Multiple images
\usepackage{multicol} % Multiple columns
\usepackage{float}
\usepackage{cleveref}
\usepackage{relsize} % bigger math symbols
\usepackage[margin=1.1in]{geometry}
\usepackage[titletoc,toc,title]{appendix}
\usepackage{enumitem}
\usepackage{etoolbox}
\usepackage{mdframed} %frame theorems
\usetikzlibrary{calc}
\numberwithin{equation}{section}

% Footnotes with symbols

\makeatletter
\def\@fnsymbol#1{\ensuremath{\ifcase#1\or \dagger\or \ddagger\or
   \mathsection\or \mathparagraph\or \|\or **\or \dagger\dagger
   \or \ddagger\ddagger \else\@ctrerr\fi}}
\makeatother

\renewcommand{\thefootnote}{\fnsymbol{footnote}}

% Cool letters 
\input{Typocaps.fd}

% Footer
\usepackage{fancyhdr}
\pagestyle{fancy}
\fancyhf{}
\cfoot{\fontsize{15pt}{15pt}\usefont{U}{Typocaps}{xl}{n} 
gigantium humeris insidentes}

% Big Pictures
\usepackage[export]{adjustbox}

% Enviroment for theorems
\newmdtheoremenv[frametitle=Teorema]{theo}{Theorem}

% Circled words
\newcommand{\circled}[2][]{%
  \tikz[baseline=(char.base)]{%
    \node[shape = circle, draw, inner sep = 1pt]
    (char) {\phantom{\ifblank{#1}{#2}{#1}}};%
    \node at (char.center) {\makebox[0pt][c]{#2}};}}
\robustify{\circled}

%Appendices in spanish
\renewcommand{\appendixname}{Ap\'endices}
\renewcommand{\appendixtocname}{Ap\'endices}
\renewcommand{\appendixpagename}{Ap\'endices}

%Zero delimiter
\newcommand{\zerodel}{.\kern-\nulldelimiterspace}

%Columns separation
\setlength{\columnsep}{1cm}

%Indentation
\setlength{\parindent}{0ex}

%Multiple References

\crefrangelabelformat{equation}{(#3#1#4--#5\crefstripprefix{#1}{#2}#6)}

\usepackage{xparse}

%Boxes

\newcommand*{\boxcolor}{blue}
\makeatletter
\renewcommand{\boxed}[1]{\textcolor{\boxcolor}{%
\tikz[baseline={([yshift=-1ex]current bounding box.center)}] \node [rectangle, minimum width=1ex,rounded corners,draw] {\normalcolor\m@th$\displaystyle#1$};}}
 \makeatother

%Constantes
\newcommand{\euler}{\mathrm{e}}
\newcommand{\im}{i}

%Lemas, teoremas, definiciones y pruebas
\newcommand{\definicion}{\textbf{Definición: }}
\newcommand{\lema}{\textbf{Lema: }}
\newcommand{\teorema}{\textbf{Teorema: }}
\newcommand{\prueba}{\textbf{Prueba: }}
\newcommand{\proposicion}{\textbf{Proposición: }}
\newcommand{\corolario}{\textbf{Corolario: }}

% Definición de las secciones y su numeración

\makeatletter
\def\@seccntformat#1{%
  \expandafter\ifx\csname c@#1\endcsname\c@section\else
  \csname the#1\endcsname\quad
  \fi}
\makeatother

\begin{document}

\begin{titlepage}
\thispagestyle{fancy}

\newcommand{\HRule}{\rule{\linewidth}{0.5mm}} % Defines a new command for the horizontal lines, change thickness here

\center % Center everything on the page
 
%----------------------------------------------------------------------------------------
%	HEADING SECTIONS
%----------------------------------------------------------------------------------------

\textsc{\LARGE Universidad Nacional Autónoma de México}\\[0.3cm] % Name of your university/college

%----------------------------------------------------------------------------------------
%	LOGO SECTION
%----------------------------------------------------------------------------------------

\includegraphics[scale=0.17]{unam}

%----------------------------------------------------------------------------------------
%	SUBHEADING SECTIONS
%----------------------------------------------------------------------------------------

\textsc{\Large Electrodinámica Clásica}\\[0.3cm] % Major heading such as course name
\textsc{\large Semestre 2016-II}\\[0.3cm] % Minor heading such as course title
\textsc{\large 10 de marzo de 2016}\\ % Minor heading such as course title

%----------------------------------------------------------------------------------------
%	TITLE SECTION
%----------------------------------------------------------------------------------------

\HRule \\[0.1cm]
{ \huge \bfseries Tarea \# 4. \\ Multipolos, Electrostática 
de Medios Macroscópicos, Dieléctricos.}\\ % Title of your document
\HRule \\[0.1cm]
 
%----------------------------------------------------------------------------------------
%	AUTHOR SECTION
%----------------------------------------------------------------------------------------
\setcounter{footnote}{0}
\center
\large
\emph{Autor:} \\ % Your name
\Large Favio \textsc{Vázquez}\footnote[1]{\href{mailto:favio.vazquez@correo.nucleares.unam.mx}{favio.vazquez@correo.nucleares.unam.mx}}

%----------------------------------------------------------------------------------------
%	COOL IMAGE SECTION
%----------------------------------------------------------------------------------------

\includegraphics[scale=0.28]{escalerasEscher}

%----------------------------------------------------------------------------------------

\vfill % Fill the rest of the page with whitespace

\end{titlepage}

% ---------------------------------------------------------------------------------------
%         HEADER
%----------------------------------------------------------------------------------------

\fancyhead[L]{Favio Vázquez}
\fancyhead[R]{\thepage}

%----------------------------------------------------------------------------------------
\setcounter{footnote}{0}
\renewcommand*{\thefootnote}{\arabic{footnote}}

\section{Problema 1. Problema 4.1 de Classical Electromagnetic Radiation
de Marion y Heald \cite{marion2}.}

Calcule los momentos multipolares $q_{lm}$ de las distribuciones de carga mostradas 
como las partes $a$ y $b$. Intente obtener resultados para todo los momentos que 
no se hacen cero válidos para todo $l$, pero en cada caso encuentre los primeros 
dos conjuntos de momentos que no se hacen cero al menos.

\begin{figure}[H]
 \center 
 \includegraphics[scale = 0.5]{problema1fig1}
 \caption{Disposición de la distribución de cargas para el problema 1.}
\end{figure}

\begin{enumerate}[label=\textbf{(\alph*)}]
\setcounter{enumi}{2}
\item Para la distribución de carga del segundo conjunto $b$ escriba la expansión 
multipolar para el potencial. Manteniendo solo los términos de orden bajo en la 
expansión, grafique el potencial en el plano $x-y$ como una función de la distancia 
desde el origen para distancias mayores a $a$.
\item Calcule directamente de la ley de Coulomb el potencial exacto para $b$ en 
el plano $x-y$. Grafíquelo como una función de la distancia y compare con el resultados
encontrado en la parte $c$.
\end{enumerate}

Divida la forma asintótica en las partes $c$ y $d$ para ver el comportamiento a 
distancias grandes más claramente.

\vspace{.3cm}

\underline{Solución:} \vspace{.3cm}

\newpage

\section{Problema 2. Problema 4.7 de Classical Electromagnetic Radiation
de Marion y Heald \cite{marion2}.}

Una distribución de carga localizada tiene la densidad de carga 

$$
\rho(\mathbf{r}) = \frac{1}{64\pi}r^2\euler^{-r}\sen^2{\theta}
$$

\begin{enumerate}[label=\textbf{(\alph*)}]
\item Haga una expansión multipolar del potencial debido a esta densidad de 
carga y determine todos los momentos multipolares que no se hacen cero. Escriba 
el potencial a grandes distancias como una expansión finita en polinomios de 
Legendre. 
\item Determine el potencial explícitamente en cualquier punto del espacio, y 
muestre que cerca del origen, correcto a $r^2$ inclusive,

$$
\Phi(\mathbf{r}) \simeq \frac{1}{4\pi\epsilon_0}\left[\frac{1}{4} 
 - \frac{r^2}{120}P_2(\cos{\theta})\right]
$$

\item Si existe en el origen un núcleo con un momento cuadrupolar $Q = 10^{-28}$ m$^2$, 
determine la magnitud de la energía de interacción, asumiendo que la unidad de 
carga en $\rho(\mathbf{r})$ arriba es la carga electrónica y que la unidad de 
longitud es el radio de Bohr del hidrógeno $a_0 = 4\pi\epsilon_0\hslash/me^2 = 
0.529 \times 10^{-10}$ m. Exprese su respuesta como una frecuencia dividida por 
la constante de Planck $h$.
\end{enumerate}

La densidad de carga en este problema es la de los estados $m = \pm 1$ del nivel 
$2p$ en el hidrógeno, mientras que la interacción cuadrupolar es del mismo orden 
que el encontrado en moléculas.

\vspace{.3cm}

\underline{Solución:} \vspace{.3cm}

\newpage

\section{Problema 3. Problema 4.8 de Classical Electromagnetic Radiation
de Marion y Heald \cite{marion2}.}

Un cascarón cilíndrico, muy largo y circular recto, de constante dieléctrica 
$\epsilon/\epsilon_0$ y radio interno y externo $a$ y $b$, respectivamente, 
es colocado en un previamente campo eléctrico uniforme $E_0$ con su eje perpendicular 
al campo. El medio adentro y afuera del cilindro tiene una constante dieléctrica de 
uno.

\begin{enumerate}[label=\textbf{(\alph*)}]
\item Determine el potencial y campo eléctrico en las tres regiones, despreciando 
efectos finales.
\item Esboce las líneas de fuerza para un caso típico de $b \simeq 2a$.
\item Discuta las formas limitantes en su solución apropiadas para un cilindro 
dieléctrico sólido en un campo uniforme, y una cavidad cilíndrica en un dieléctrico 
uniforme.
\end{enumerate}

\vspace{.3cm}

\underline{Solución:} \vspace{.3cm}

\newpage

\section{Problema 4. Problema 4.9 de Classical Electromagnetic Radiation
de Marion y Heald \cite{marion2}.}

Una carga puntual $q$ es colocada en el espacio libre a una distancia $d$ del centro 
de una esfera dieléctrica de radio $a$ ($a < d)$ y constante dieléctrica 
$\epsilon/\epsilon_0$. 

\begin{enumerate}[label=\textbf{(\alph*)}]
\item Encuentre el potencial en todos los puntos del espacio como una expansión 
en armónicos esféricos.
\item Calcule las componentes rectangulares del campo eléctrico cerca del 
centro de la esfera.
\item Verifique que, en el límite $\epsilon/\epsilon_0 \rightarrow \infty$, tu 
resultado es el mismo que para la esfera conductora.
\end{enumerate}

\vspace{.3cm}

\underline{Solución:} \vspace{.3cm}

\newpage

\section{Problema 5. Problema 4.13 de Classical Electromagnetic Radiation
de Marion y Heald \cite{marion2}.}

Dos superficies cilíndricas conductoras, largas y coaxiales, de radios $a$ y $b$ 
son bajadas verticalmente a un líquido dieléctrico. Si el líquido sube una altura 
promedio de $h$ entre los electrodos cuando una diferencia de potencial $V$ es 
establecida entre ellos, muestre que la susceptibilidad del líquido es 

$$
\chi_e = \frac{(b^2 - a^2)\rho g h \ln{(b/a)}}{\epsilon_0 V^2}
$$

donde $\rho$ es la densidad del líquido, $g$ es la aceleración debida a la gravedad, 
y la susceptibilidad del aire es depreciada.

\vspace{.3cm}

\underline{Solución:} \vspace{.3cm}

\newpage

\begin{thebibliography}{10}
\bibitem{marion2}
J. Marion, M. Heald, \emph{Classical Electromagnetic Radiation}, 2da edición, Academic 
Press, 1965.
\end{thebibliography}

\end{document}