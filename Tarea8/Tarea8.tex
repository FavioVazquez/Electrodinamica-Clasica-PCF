\documentclass[a4paper,11pt]{article}
\usepackage[utf8]{inputenc}
\usepackage[spanish]{babel}
\usepackage[affil-it]{authblk}
\usepackage{enumerate}
\usepackage{graphicx}
\usepackage{listings}
\usepackage{hyperref}
\usepackage{amsmath}
\usepackage{amssymb}
\usepackage{cancel}
\usepackage[usenames, dvipsnames]{color}
\usepackage{tikz}
\usepackage[labelfont=bf]{caption}
\usepackage{subcaption} %Multiple images
\usepackage{multicol} % Multiple columns
\usepackage{float}
\usepackage{cleveref}
\usepackage{relsize} % bigger math symbols
\usepackage[margin=1.1in]{geometry}
\usepackage[titletoc,toc,title]{appendix}
\usepackage{enumitem}
\usepackage{etoolbox}
\usepackage{mdframed} %frame theorems
\usetikzlibrary{calc}
\numberwithin{equation}{section}

% Footnotes with symbols

\makeatletter
\def\@fnsymbol#1{\ensuremath{\ifcase#1\or \dagger\or \ddagger\or
   \mathsection\or \mathparagraph\or \|\or **\or \dagger\dagger
   \or \ddagger\ddagger \else\@ctrerr\fi}}
\makeatother

\renewcommand{\thefootnote}{\fnsymbol{footnote}}

%Styling for code
\definecolor{codegreen}{rgb}{0,0.6,0}
\definecolor{codegray}{rgb}{0.5,0.5,0.5}
\definecolor{codepurple}{rgb}{0.58,0,0.82}
\definecolor{backcolour}{rgb}{0.95,0.95,0.92}
 
\lstdefinestyle{mystyle}{
    backgroundcolor=\color{backcolour},   
    commentstyle=\color{codegreen},
    keywordstyle=\color{magenta},
    numberstyle=\tiny\color{codegray},
    stringstyle=\color{codepurple},
    basicstyle=\footnotesize,
    breakatwhitespace=false,         
    breaklines=true,                 
    captionpos=b,                    
    keepspaces=true,                 
    numbers=left,                    
    numbersep=5pt,                  
    showspaces=false,                
    showstringspaces=false,
    showtabs=false,                  
    tabsize=2
}
 
\lstset{style=mystyle}

% Cool letters 
\input{Typocaps.fd}

% Footer
\usepackage{fancyhdr}
\pagestyle{fancy}
\fancyhf{}
\cfoot{\fontsize{15pt}{15pt}\usefont{U}{Typocaps}{xl}{n} 
gigantium humeris insidentes}

% Big Pictures
\usepackage[export]{adjustbox}

% Enviroment for theorems
\newmdtheoremenv[frametitle=Teorema]{theo}{Theorem}

% Circled words
\newcommand{\circled}[2][]{%
  \tikz[baseline=(char.base)]{%
    \node[shape = circle, draw, inner sep = 1pt]
    (char) {\phantom{\ifblank{#1}{#2}{#1}}};%
    \node at (char.center) {\makebox[0pt][c]{#2}};}}
\robustify{\circled}

%Appendices in spanish
\renewcommand{\appendixname}{Ap\'endices}
\renewcommand{\appendixtocname}{Ap\'endices}
\renewcommand{\appendixpagename}{Ap\'endices}

%Zero delimiter
\newcommand{\zerodel}{.\kern-\nulldelimiterspace}

%Columns separation
\setlength{\columnsep}{1cm}

%Indentation
\setlength{\parindent}{0ex}

%Multiple References

\crefrangelabelformat{equation}{(#3#1#4--#5\crefstripprefix{#1}{#2}#6)}

\usepackage{xparse}

%Boxes

\newcommand*{\boxcolor}{blue}
\makeatletter
\renewcommand{\boxed}[1]{\textcolor{\boxcolor}{%
\tikz[baseline={([yshift=-1ex]current bounding box.center)}] \node [rectangle, minimum width=1ex,rounded corners,draw] {\normalcolor\m@th$\displaystyle#1$};}}
 \makeatother

%Constantes
\newcommand{\euler}{\mathrm{e}}
\newcommand{\im}{i}

%Lemas, teoremas, definiciones y pruebas
\newcommand{\definicion}{\textbf{Definición: }}
\newcommand{\lema}{\textbf{Lema: }}
\newcommand{\teorema}{\textbf{Teorema: }}
\newcommand{\prueba}{\textbf{Prueba: }}
\newcommand{\proposicion}{\textbf{Proposición: }}
\newcommand{\corolario}{\textbf{Corolario: }}

% Definición de las secciones y su numeración

\makeatletter
\def\@seccntformat#1{%
  \expandafter\ifx\csname c@#1\endcsname\c@section\else
  \csname the#1\endcsname\quad
  \fi}
\makeatother

\begin{document}

\begin{titlepage}
\thispagestyle{fancy}

\newcommand{\HRule}{\rule{\linewidth}{0.5mm}} % Defines a new command for the horizontal lines, change thickness here

\center % Center everything on the page
 
%----------------------------------------------------------------------------------------
%	HEADING SECTIONS
%----------------------------------------------------------------------------------------

\textsc{\LARGE Universidad Nacional Autónoma de México}\\[0.3cm] % Name of your university/college

%----------------------------------------------------------------------------------------
%	LOGO SECTION
%----------------------------------------------------------------------------------------

\includegraphics[scale=0.17]{unam}

%----------------------------------------------------------------------------------------
%	SUBHEADING SECTIONS
%----------------------------------------------------------------------------------------

\textsc{\Large Electrodinámica Clásica}\\[0.3cm] % Major heading such as course name
\textsc{\large Semestre 2016-II}\\[0.3cm] % Minor heading such as course title
\textsc{\large 7 de abril de 2016}\\ % Minor heading such as course title

%----------------------------------------------------------------------------------------
%	TITLE SECTION
%----------------------------------------------------------------------------------------

\HRule \\[0.1cm]
{ \huge \bfseries Tarea \# 8. \\ Deducción de los campos eléctricos y magnéticos de velocidad y 
aceleración a partir de los potenciales de Lienard-Wiechert.}\\ % Title of your document
\HRule \\[0.1cm]
 
%----------------------------------------------------------------------------------------
%	AUTHOR SECTION
%----------------------------------------------------------------------------------------
\setcounter{footnote}{0}
\center
\large
\emph{Autor:} \\ % Your name
\Large Favio \textsc{Vázquez}\footnote[1]{\href{mailto:favio.vazquez@correo.nucleares.unam.mx}{favio.vazquez@correo.nucleares.unam.mx}}
\\[0.7cm]
%----------------------------------------------------------------------------------------
%	COOL IMAGE SECTION
%----------------------------------------------------------------------------------------

\includegraphics[scale=0.55]{escherEspiral}

%----------------------------------------------------------------------------------------

\vfill % Fill the rest of the page with whitespace

\end{titlepage}

% ---------------------------------------------------------------------------------------
%         HEADER
%----------------------------------------------------------------------------------------

\fancyhead[L]{Favio Vázquez}
\fancyhead[R]{\thepage}

%----------------------------------------------------------------------------------------
\setcounter{footnote}{0}
\renewcommand*{\thefootnote}{\arabic{footnote}}
%----------------------------------------------------------------------------------------

%----------------------------------------------------------------------------------------
%%			BEGIN DOCUMENT
%----------------------------------------------------------------------------------------

\section{Diferenciación de los potenciales de Lienard-Wiechert}

Esta deducción se hizo siguiendo los resultados de las secciones 6.2 y 6.3 del 
libro de Schwartz \cite{schwartz}. Los potenciales de Lienard-Wiechert podemos 
escribirlos como 

\begin{equation}
 \phi(\mathbf{r},t) = \frac{q}{|\mathbf{r}-\mathbf{r}'(t')\left[1 - 
 \frac{v'(t')\cdot \hat{\mathbf{\epsilon}}}{c}\right]},
\end{equation}

y 

\begin{equation}
 \mathbf{A}(\mathbf{r},t) = \frac{q\mathbf{v}'(t')}{c|\mathbf{r}-\mathbf{r}'(t')\left[1 - 
 \frac{v'(t')\cdot \hat{\mathbf{\epsilon}}}{c}\right]},
\end{equation}

donde $q$ es la carga de la partícula, $\mathbf{v}(t)$ y $\mathbf{\epsilon}$ 
es un vector unitario de $\mathbf{r}'(t')$ a $\mathbf{r}$, 

\begin{equation}
 \mathbf{\epsilon}' = \frac{\mathbf{r}-\mathbf{r}'(t')}{|\mathbf{r}-\mathbf{r}'(t')|}.
\end{equation}

Podemos pasar ahora a calcular ahora los campos eléctricos y magnéticos debidos a 
nuestras pequeñas cargas en movimiento. Aunque en el régimen no relativista, nos 
interesan situaciones donde $v \ll c$, trabajaremos por el momento manteniendo 
todos los órdenes en $v/c$. Nuestro trabajo se convierte entonces en diferenciar 
estos potenciales, para encontrar 

\begin{equation}
 \mathbf{B}(\mathbf{r},t) = 
 \pmb{\nabla} \times \frac{q\mathbf{v}'(t')}{c|\mathbf{r}-\mathbf{r}'(t')\left[1 - 
 \frac{v'(t')\cdot \hat{\mathbf{\epsilon}}}{c}\right]},
\end{equation}

\begin{equation}
 \mathbf{E}(\mathbf{r},t) = - \pmb{\nabla}  \frac{q}{|\mathbf{r}-\mathbf{r}'(t')\left[1 - 
 \frac{v'(t')\cdot \hat{\mathbf{\epsilon}}}{c}\right]} - \frac{1}{c} 
 \frac{\partial}{\partial t}\left[\frac{q\mathbf{v}'(t')}{c|\mathbf{r}-\mathbf{r}'(t')\left[1 - 
 \frac{v'(t')\cdot \hat{\mathbf{\epsilon}}}{c}\right]} \right].
\end{equation}

La dificultad de estas derivaciones yace en la compleja dependencia implícita 
de los términos primados en $\mathbf{r}$ y $t$. Definiendo el tiempo retardado 
como 








\begin{thebibliography}{10}
\bibitem{schwartz}
 M. Schwartz, \emph{Principles of electrodynamics}, Dover Publications, Inc. 1972.
\end{thebibliography}

\end{document}