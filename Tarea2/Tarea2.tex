\documentclass[a4paper,10pt]{article}
\usepackage[utf8]{inputenc}
\usepackage[spanish]{babel}
\usepackage[affil-it]{authblk}
\usepackage{enumerate}
\usepackage{graphicx}
\usepackage{hyperref}
\usepackage{amsmath}
\usepackage{amssymb}
\usepackage{cancel}
\usepackage[usenames, dvipsnames]{color}
\usepackage{tikz}
\usepackage[labelfont=bf]{caption}
\usepackage{subcaption} %Multiple images
\usepackage{multicol} % Multiple columns
\usepackage{float}
\usepackage{cleveref}
 \usepackage{relsize} % bigger math symbols
\usepackage[margin=1.4in]{geometry}
\usepackage[titletoc,toc,title]{appendix}
\usepackage{enumitem}
\usepackage{etoolbox}
\usepackage{mdframed} %frame theorems
\usetikzlibrary{calc}
\numberwithin{equation}{section}

% Enviroment for theorems
\newmdtheoremenv[frametitle=Teorema]{theo}{Theorem}

% Circled words
\newcommand{\circled}[2][]{%
  \tikz[baseline=(char.base)]{%
    \node[shape = circle, draw, inner sep = 1pt]
    (char) {\phantom{\ifblank{#1}{#2}{#1}}};%
    \node at (char.center) {\makebox[0pt][c]{#2}};}}
\robustify{\circled}

%Appendices in spanish
\renewcommand{\appendixname}{Ap\'endices}
\renewcommand{\appendixtocname}{Ap\'endices}
\renewcommand{\appendixpagename}{Ap\'endices}

%Zero delimiter
\newcommand{\zerodel}{.\kern-\nulldelimiterspace}

%Columns separation
\setlength{\columnsep}{1cm}

%Indentation
\setlength{\parindent}{0ex}

%Multiple References

\crefrangelabelformat{equation}{(#3#1#4--#5\crefstripprefix{#1}{#2}#6)}

\usepackage{xparse}

%Boxes

\newcommand*{\boxcolor}{blue}
\makeatletter
\renewcommand{\boxed}[1]{\textcolor{\boxcolor}{%
\tikz[baseline={([yshift=-1ex]current bounding box.center)}] \node [rectangle, minimum width=1ex,rounded corners,draw] {\normalcolor\m@th$\displaystyle#1$};}}
 \makeatother

%Constantes
\newcommand{\euler}{\mathrm{e}}
\newcommand{\im}{i}

%Lemas, teoremas, definiciones y pruebas
\newcommand{\definicion}{\textbf{Definición: }}
\newcommand{\lema}{\textbf{Lema: }}
\newcommand{\teorema}{\textbf{Teorema: }}
\newcommand{\prueba}{\textbf{Prueba: }}
\newcommand{\proposicion}{\textbf{Proposición: }}
\newcommand{\corolario}{\textbf{Corolario: }}

% Definición de las secciones y su numeración

\makeatletter
\def\@seccntformat#1{%
  \expandafter\ifx\csname c@#1\endcsname\c@section\else
  \csname the#1\endcsname\quad
  \fi}
\makeatother

%opening
\title{Electrodinámica Clásica. Tarea \# 2}
\author{Favio Vázquez\thanks{Correo: favio.vazquezp@gmail.com}}\affil{Instituto de Ciencias Nucleares. Universidad Nacional Autónoma de México.}
\date{}
\begin{document}

\makeatletter
\def\@maketitle{%
  \newpage
  \null
  \vskip 2em%
  \begin{center}%
  \let \footnote \thanks
    {\Large\bfseries \@title \par}%
    \vskip 1.5em%
    {\normalsize
      \lineskip .5em%
      \begin{tabular}[t]{c}%
        \@author
      \end{tabular}\par}%
    \vskip 1em%
    {\normalsize \@date}%
  \end{center}%
  \par
  \vskip 1.5em}
\makeatother

\maketitle

\section{Problema 1. Problema 2.1 de Classical Electrodynamics (tanto en la 2da 
como en la 3ra edición) de Jackson \cite{jackson2,jackson3}.}

Una carga puntual $q$ es llevada a una posición a una distancia $d$ desde un 
plano conductor infinito que está a un potencial cero. Usando el método de imágenes, 
encuentre: 

\begin{enumerate}[label=\textbf{(\alph*)}]
 \item la densidad de carga superficial inducida en el plano, y grafíquela;
 \item la fuerza entre el plano y la carga usando la ley de Coulomb para la fuerza 
 entre la carga y su imagen;
 \item la fuerza total actuando en el plano integrando $\sigma^2/2\epsilon_0$ 
 sobre todo el plano;
 \item el trabajo necesario para remover la carga $q$ de su posición al infinito;
 \item la energía potencial entre la carga $q$ y su imagen [compare la respuesta 
 con la de la parte (d) y discuta].
 \item Encuentre la respuesta a la parte (d) en eV para un electrón originalmente 
 a un angstrom de la superficie.
\end{enumerate}

\vspace{.3cm}

\underline{Solución:} \vspace{.3cm}

\section{Problema 2. Problema 2.2 de Classical Electrodynamics (tanto en la 2da 
como en la 3ra edición) de Jackson \cite{jackson2,jackson3}.}

Usando el método de imágenes, discuta el problema de una carga puntual $q$ 
\emph{adentro} de una esfera hueca, conectada a tierra, conductora de radio interno
$a$. Encuentre 

\begin{enumerate}[label=\textbf{(\alph*)}]
 \item el potencial adentro de la esfera;
 \item la densidad de carga superficial inducida;
 \item la magnitud y dirección de la fuerza actuando sobre $q$.
 \item ¿Hay algún cambio en la solución si la esfera es mantenida a un potencial 
 fijo $V$? ¿y si la esfera tiene una carga total $Q$ en sus superficies internas y 
 externas?
\end{enumerate}

\vspace{.3cm}

\underline{Solución:} \vspace{.3cm}

\section{Problema 3. Problema 2.3 de Classical Electrodynamics (2da edición) de Jackson 
\cite{jackson2} y 2.7 (3ra edición) de Jackson \cite{jackson3}.}

Considera un problema de potencial en el medio espacio definido por $z \geq 0$, con 
condiciones de frontera de Dirichlet sobre el plano $z = 0$ (y en infinito).

\begin{enumerate}[label=\textbf{(\alph*)}]
 \item Escribe la función de Green apropiada $G(\mathbf{x},\mathbf{x}')$.
 \item Si el potencial $z = 0$ es especificado por $\Phi = V$ adentro de un círculo 
 de radio $a$ centrado en el origen, y $\Phi = 0$ afuera del círculo, encuentre 
 una expresión integral para el potencial en el punto $P$ especificado en términos 
 de coordenadas cilíndricas $(\rho,\phi,z)$.
 \item Muestre que, a lo largo del eje del círculo $(\rho = 0)$, el potencial está 
 dado por 
 $$
 \Phi = V\left(1 - \frac{z}{\sqrt{a^2+z^2}}\right)
 $$
 \item Muestre que para distancias grandes $(\rho^2 + z^2 \gg a^2)$ el potencial 
 puede ser expandido en una serie de potencias en $(\rho^2 + z^2)^{-1}$, y que 
 los términos más importantes son 
 $$
 \Phi = \frac{Va^2}{2}\frac{z}{(\rho^2 + z^2)^{3/2}}\left[1 - 
 \frac{3a^2}{4(\rho^2 + z^2)} + \frac{5(3\rho^2a^2+a^4)}{8\rho^2 + z^2)^2} 
 + \dots \right]
 $$
\end{enumerate}

Verifica que los resultados de las partes (c) y (d) son consistentes el uno 
con el otro en su rango común de validez.

\vspace{.3cm}

\underline{Solución:} \vspace{.3cm}

\section{Problema 4. Problema 2.5 de Classical Electrodynamics (2da edición) de Jackson 
\cite{jackson2} y 2.9 (3ra edición) de Jackson \cite{jackson3}.}

Una concha conductora, aislada y esférica de radio $a$ está en un campo eléctrico 
uniforme $E_0$. Si la esfera es cortada en dos hemisferios por un plano perpendicular 
al campo, encuentre la fuerza requerida para prevenir que los hemisferios se 
separen 

\begin{enumerate}[label=\textbf{(\alph*)}]
 \item si la concha no tiene carga;
 \item si la carga total en la concha es $Q$.
\end{enumerate}

\vspace{.3cm}

\underline{Solución:} \vspace{.3cm}

\section{Problema 5. Problema 2.6 de Classical Electrodynamics (2da edición) de Jackson 
\cite{jackson2} y 2.10 (3ra edición) de Jackson \cite{jackson3}.}

Un capacitor de placas paralelas grande está hecho de dos láminas conductoras 
planas con una separación $D$, una de ellas tiene tiene un bulto semiesférico 
de radio $a$ en su superficie interna $(D \gg a)$. El conductor con el bulto 
es puesto a un potencial cero, y el otro conductor es a un potencial tal que, 
lejos del bulto, el campo eléctrico entre las placas es $E_0$.

\begin{enumerate}[label=\textbf{(\alph*)}]
 \item Calcule la densidad de carga superficial en un punto arbitrario del 
 plano y sobre el bulto, y esboce su comportamiento como una función de la distancia 
 (o ángulo).
 \item Muestre que la carga total en el bulto tiene la magnitud $3\pi\epsilon_0E_0a^2$.
 \item Si, en cambio de tener la otra lámina a un potencial diferente, una carga 
 puntual $q$ es colocada directamente arriba del bulto semiesférico a una distancia 
 $d$ de su centro, muestre que la carga inducida sobre el bulto es 
 $$
 q' = -q\left[1 - \frac{d^2 - a^2}{d\sqrt{d^2 + a^2}} \right]
 $$
\end{enumerate}

\vspace{.3cm}

\underline{Solución:} \vspace{.3cm}

\begin{thebibliography}{10}
\bibitem{jackson2}
J. Jackson, \emph{Classical Electrodynamics}, 2da edición. John Wiley and Sons, Inc. 
1975.
\bibitem{jackson3}
J. Jackson, \emph{Classical Electrodynamics}, 3ra edición. John Wiley and Sons, Inc. 
1999.
\end{thebibliography}

\end{document}